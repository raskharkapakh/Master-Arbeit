\documentclass{article}
\usepackage[utf8]{inputenc}
\usepackage{cite}
\usepackage[T1]{fontenc}
\usepackage{geometry}
\usepackage{amsmath}
\usepackage{amssymb}
\usepackage{comment}
\usepackage[round]{natbib}

%\geometry{
%    a4paper,
%    total={170mm, 257mm},
%    left=20mm,
%    top=10mm,
%}

\title{Master Thesis}
\author{Gaspard Ulysse Fragnière}
\date{August 2022}

\begin{document}

\maketitle

\section{State-of-the-art}

The problem of Acoustical Source Localization (ASL) is an important problem. It was many applications suach as smart assistant (e.g. Google Home, Alexa, ...), industrial applications, \textbf{TODO: add more?}. Traditionnaly this problem is tackled with methods based on the physics of sound propagation (e.g. TDoA, beamforming) or with statistcal inference (e.g.Sparse Bayesian Learning). 

The recent success of Deep Learning (DL) based method in other field of research (e.g. Computer Vision) led to believe that Deep Neural Networks (DNN) based approaches could provide state-of-the-art result in solving the ASL problem. \cite{castellini2021neural}, \cite{kujawski2019deep}, \cite{lee2021deep}, \cite{ma2019phased}, \cite{pinto2021deconvoluting} and \cite{xu2021deep} propose state-of-the-arts DL-based methods for Source Characterization. 


A common issue faced while implementing DL-based methods is that significant quantities of well structures data are required. In the litterature, the data has been obtained using the following approaches:

\begin{itemize}
    \item \textbf{Real Mesurement}: To create the different samples of such a dataset, sounds emitted with a loudspeaker or human voices are recorded in a real acoustic enviromnent. Eventhough such a method allows for the creation of perfectly realistic samples, it does not come without any issue. Indeed, it is very tedious and time consuming to record in different environment. Additionally, all the environment for measurement need to physically exists, which limits the quantity of possible samples. Moreover, to build a high quality data set, expensive equipment is required to have an accurate groundtruth (i.e. precisely identify the location of the sources). In the literature, \cite{he2018deep} and \cite{ferguson2018sound} have used such an approach.
    \item \textbf{Synthetic Data}: The sounds used are artificial (i.e. white noise, sine wave). The room acoustic is also simulated. Indeed the dry sound is convolved with a simulated Room Impulse Response (RIR) to mimic the effect of room acoustics (e.g. reverberation). Compared to real measurement, this approach allows sample in more diverse environment. Indeed RIR for rooms of arbitrary size, different source position as well as different dry signals can be used for the training. The issue with such a method is the important amount of time and storage required for the creation of the datasets. E.g. \cite{chakrabarty2017broadband}, \cite{perotin2018crnn} and \cite{adavanne2018direction} created their datasets in this way.
    \item \textbf{Semi-synthetic data}: The creation of such a dataset is similar the creation of synthetic dataset. The difference lies in the fact that the dry sound source used and the RIR are measured and not simulated. Then, the samples of such a dataset are generated by convolving dry sounds with RIR. This method is not the best suited, since it is very time-consuming to generate a data set with enough samples for training a DL-based algorithm. Indeed, measuring all the RIR lead to the issues faced with real measurement. \cite{takeda2016sound} use such an approach for to obtain their data.
\end{itemize}

Moreover it is to be noted that none of these methods are suitable for online data generation. Indeed, any of the above mentionned method do not allow for creating random sample while training DL-based algorithm. To use such datasets for training, they need to be fully created (and stored) before any training can occur. 

\subsection{DL-based data generation}

In the past years, DL-based approaches have shown to be able to learn and realistically reproduce very complicated data structures (e.g. generation of pictures of faces in the field of Computer Vision). Those breakthroughs lead to believe that similar data generation methods could be used to fix the above-mentionned issues (e.g. offline training, lack of variance in the different samples, ...).

Moreover, it is relevant to note that the data used for source characterization in \cite{castellini2021neural}, \cite{lee2021deep}, \cite{ma2019phased}, \cite{xu2021deep} is the Cross Power Spectra (CPS), i.e. a direct representation of the signals received in the array of microphone. Indeed those approaches do not use direct recording of microphone input but instead features extracted from the raw data. This is crucial because it means that recording, simulating or generating raw microphone data is no longer necessary, if features (e.g. CPS) could be generated directly. We therefore need to identify what acoustic quantities:
\begin{itemize}
    \item have already been generated using a DL approach
    \item are potential feature for a Source Characterization Algorithm.
\end{itemize}


\subsubsection{Generation of Signal}

\cite{neekhara2019expediting}, \cite{NEURIPS2019_6804c9bc}, \cite{engel2019gansynth} use Generative Adversial Network (GAN) to generate realistic audio waveform. \cite{neekhara2019expediting} and \cite{NEURIPS2019_6804c9bc} specifically focus on the generation of audio waveform conditioned on a spectogram (cGAN). On the other hand, \cite{engel2019gansynth} design a GAN to generate realistic audio waveform of single music notes played by an instrument. The data generated in those approaches is single-channel data, but maybe it could be extended to multi-channel to simulate the different signals recorded in an array of microphone. It is relevant to note that the GAN designed by \cite{neekhara2019expediting} is the one implemented in \cite{vargas2021improved} in order to compare the accuracy of a network for single source DoA estimation when trained with different sound classes.

\subsubsection{Generation of Impulse Response}

\cite{papayiannis2019data} introduce a GAN approach to generate artificial Acoustical Impulse Response (AIR) of different environment in order to generate data for a NN used for classification of acoustic environment.

\cite{ratnarajah2021fast} proposes a fast method (NN-based) for generating Room Impulse Response (RIR). The input paramaters of the networks used for creating the IR are the desired dimensions of the rectangular room, listener position, speaker position and reverberation time ($T_{60}$).

\textbf{TODO: is it worth mentionning both papers ?}

This is relevant for a problem at hand because if we are to be able to generate impulse responses with known source and listener position, we could simply convolve them with the dry source sounds. This way, we could generate raw microphone signal and use them to train a DL-based algorithm for source characterization.


\subsubsection{Generation of potential NN feature}

\cite{bianco2020semi} proposes an approach to generate another acoustic feature: the phase of the relative transfer function (RTF) between two microphones. In this paper a Variational Auto Encoder (VAE) is designed to simultaneously generate phases of RTF and classifying them by their Direction of Arrival (DoA).

\cite{gerstoft2020parametric} use a GAN to generate Sample Cross Spectra Matrices (CSM). for a given DoA. In their approach, the GAN is trained with data only coming from one DoA, making it unable to generate sample for different DoA. This approach could be extended by creating a conditional Generative Adversial Network (cGAN) taking as input the DoA. Such a GAN would receive a DoA as input and use it to produce a CSM corresponding to the received DoA.

\subsubsection{Other possible approaches to generate the data}

In \cite{hubner2021efficient} introduce a low complexity model-based method for generating samples of microphones phases. This method proposed is not based on DL. Indeed, it is based on a statistical noise model, a deterministic direct-path model for the point source, and a statistical model. The claim of this paper is that the low complexity of the proposed  model makes it suited for online training data generation. 

\cite{vera2021acoustic} introduce a CNN for denoising (i.e. removing the effects of reverberation and multipath effects) on the Generalization Cross Correlation (GCC) matrix of an array of microphone. More specifically than a CNN, the network used has a encoder-decoder structure. This means that a possible approach to create the data we want, would be to attempt to invert network proposed. With this we could realistically add noise to GCC matrices and hence making it suitable for training.


% Bibliography
%\addbibresource{mybib.bib}
\bibliography{../mybib}
\bibliographystyle{plainnat} 


\end{document}